%%%%%%%%%%%%%%%%%%%%%%%%%%%%%%%%%%%%%%%%%%%%%%%%%%%%%%%%%%%%%%%%%%%%%
%
%  This is a sample LaTeX input file for your contribution to 
%  the M&C2019 topical meeting.
%
%  Please use it as a template for your full paper 
%    Accompanying/related file(s) include: 
%       1. Document class/format file: mandc.cls
%       2. Sample Postscript Figure:   figure.pdf
%       3. A PDF file showing the desired appearance: mandc2019_template.pdf
%       4. cites.sty and citesort.sty that might be needed by some users 
%    Direct questions about these files to: palmert@engr.orst.edu
%											mark.dehart@inl.gov
%
%    Notes: 
%      (1) You can use the "dvips" utility to convert .dvi 
%          files to PostScript.  Then, use either Acrobat 
%          Distiller or "ps2pdf" to convert to PDF format. 
%      (2) Different versions of LaTeX have been observed to 
%          shift the page down, causing improper margins.
%          If this occurs, adjust the "topmargin" value in the
%          physor2018.cls file to achieve the proper margins. 
%
%%%%%%%%%%%%%%%%%%%%%%%%%%%%%%%%%%%%%%%%%%%%%%%%%%%%%%%%%%%%%%%%%%%%%


%%%%%%%%%%%%%%%%%%%%%%%%%%%%%%%%%%%%%%%%%%%%%%%%%%%%%%%%%%%%%%%%%%%%%
\documentclass[letterpaper]{mandc2019}
%
%  various packages that you may wish to activate for usage 
\usepackage{tabls}
\usepackage{cites}
\usepackage{epsf}
\usepackage{appendix}
\usepackage{ragged2e}
\usepackage[top=1in, bottom=1.in, left=1.in, right=1.in]{geometry}
\usepackage{enumitem}
\setlist[itemize]{leftmargin=*}
\usepackage{caption}
\captionsetup{width=1.0\textwidth,font={bf,normalsize},skip=0.3cm,within=none,justification=centering}

%\usepackage[justification=centering]{caption}

%
% Define title...
%
\title{ MESHED SHUTDOWN DOSE RATE WORKFLOW FOR  \\
  HIGH ENERGY ACCELERATOR-BASED NUCLEAR SYSTEMS}
%
% ...and authors
%
\author{%
  % FIRST AUTHORS 
  %
  \textbf{Nancy Granda Duarte$^1$, Paul Wilson$^2$} \\%\footnote{Footnote, if necessary, in Times New Roman font and font size 10} \\
  $^1$University of Wisconsin - Madison  \\
  1500 Engineering Drive \\ 
\\
  $^2$University of Wisconsin - Madison  \\ 
    Engineering Drive \\ 
\\
  \url{grandaduarte@wisc.edu}, \url{paul.wilson@wisc.edu}
}
%
% Insert authors' names and short version of title in lines below
%
\newcommand{\authorHead}      % Author's names here use et al. if more than 3
           {Nancy Granda Duarte}  
\newcommand{\shortTitle}      % Short title here (Shorten to fit all into a single line)
           {Meshed Shutdown Dose Rate Workflow for High Energy Nuclear Systems}  
%%%%%%%%%%%%%%%%%%%%%%%%%%%%%%%%%%%%%%%%%%%%%%%%%%%%%%%%%%%%%%%%%%%%%
%
%   BEGIN DOCUMENT
%
%%%%%%%%%%%%%%%%%%%%%%%%%%%%%%%%%%%%%%%%%%%%%%%%%%%%%%%%%%%%%%%%%%%%%
\begin{document}
\maketitle
\justify 

\begin{abstract}
  Use 8.5$\times$11 paper size, with 1'' margins on all sides.  A required 200-250 
  word abstract starts on this line.  Leave two blank lines before ``ABSTRACT''
  and one after.  Use 11 point Times New Roman here and single 
  spacing. The abstract is a very brief summary highlighting main 
  accomplishments, what is new, and how it relates to the state-of-the-art.
\end{abstract}
\keywords{accelerator, .... (three to five words)}

\section{INTRODUCTION} 
In high energy nuclear systems, high energy neutron irradiation of the system 
components results in the production of radioactive materials which decay to 
produce photons. These photons persist even after shutdown of the 
system. The biological dose rate from these photons must be accurately 
quatified in order to develop an appropiate procedure for maintance and operation. 
Furthermore when components reach their end of life, they must be 
replaced and supporting documentation is needed in order to store the 
component safely. A radionuclide inventory and dose rate analysis is needed. 

The Spallation Neutron Source (SNS) facility requires such quantities. 
The current workflow  used at SNS to calculate the dose rate is Rigorous 
Two-Step method facility requires such quantities. 
The current workflow  used at SNS to calculate the dose rate is Rigorous 
Two-Step (R2S) method. The R2S method entails two separate transport steps 
coupled by an activation calculation. The workflow starts with a neutron 
transport, then an activation calculation is done using results 
from the neutron transport to calculate the  
photon source to be used in a photon transport calculation. 

This work has been done at SNS using geometry cells. 
This implementations supports a workflow involving CAD geomtries  and 
superimposed cartesian meshes. 
The workflow utilizes CAD geometries for the neutron and photon transport. 


There are four types of reference styles: journal paper\cite{journal}, 
proceedings paper\cite{proc_paper}, book\cite{book}, and website\cite{website}.
References to websites are discouraged, but acceptable if absolutely necessary. It 
is the author’s responsibility to check links in the PDF file of your paper. 
All references should be cited in the text in numerical order, in order of 
appearance as [5-7].

The page limit for M\&C 2019 papers is 12 pages. It is strongly recommended that 
the length of any final paper should not exceed 12 pages.

\section{IMPLEMENTATION} 
\label{sec:first}

\subsection{Overview of the Original Workflow}

First a native MCNPX geometry representation is created, materials are assigned, 
and a neutron source is defined to use in the neutron trasnport. Tallies are used 
to collect the flux and the radioisotopic production and destruction rates on a 
geometric cell. The flux is collected to energies up to 25 MeV. The 
radionuclide production and destruction rates are collected in order to mediate 
the lack of nuclear data above the 150 MeV range in some cases. 
The flux and radionuclide information are read from the mcnp output file by 
the activation script. The activation script writes a CINDER90 input file 
and run CINDER. The CINDER output is read by a photon source script to create 
a photon source SDEF card. The photon source is used in a photon transport 
calculation to obtain the photon flux and the dose  rate. in a per cell base. 

\subsection{Overview of the Proposed Workflow}

Start with a CAD representation of the problem geometry that has material 
assigned to it. DAG-MCP6 - version of MCNP5 integrated with the Direct Accelerated 
Geometry for Monte Carlo (DAGMC) toolkit - is used to perform a 
neutron transport calculation directly on the CAD geometry. 
Energy discretized neutron fluxes are tallied on superimposed cartesian mesh 
as well as the radionuclide production and destruction rate are tallied 
on a mesh. 
The meshed flux and radionuclide information is then read by a modified 
activation script which writes a CINDER90 input file and runs CINDER90. 
Another script is used to read CINDER output and create a meshed photon source. 
An ad-hoc subroutine allows DAG-MCNP to use the meshed photon source instead 
of a source definition (SDEF) card. Finally a flux tally can be used 
with flux-to-dose conversion factors to estimate the biological dose rate. 

\subsection{Radionuclide tally }

The original radionuclide tally was created as a patch for MCNPX to 
collect radionuclide information directly in the MCNP output file 
instead of having to post-process a htape file. 
The original tally makes use of the physics model within MCNP 
in order to obtain results and they were recorded in a each 
cell requested in the input. 
Modification were made to the patch in order to collect information in 
a superimposed mesh provided in the RNUCS card. 

\subsection{Activation script}
The original activation script takes in the flux and radionuclide per cell and 
writes an input script and run an activation calculation. 
This script is modified to be able to take in flux and radionuclide information 
on a mesh and run an activation calculation. 

\subsection{Photon Source Sampling and script}
The proposed workflow produces a Cartesian mesh-based photon source therefore 
the workflow employs an ad-hoc MCNP source subroutine in order to provide a
way for sampling from the mesh-based source. 


\SECTION{Progress and Preliminary Results}

\subsection{Benchmark}
The Spallation neutron source seen Figure ... was used to test the original 
workflow and the a  workflow with the use of the DAGMC toolkit. 

The photon release rate was recorded for an original run with MCNPX and with 
DAG-MCNP6. 





\begin{figure}[!htb]
  \centering
  \includegraphics[scale=0.60]{./Figures/figure.pdf}
  \caption{SCALE/TRITON-NEWT Model of BWR Assembly in Order to Show and Example of a Figure and a Multi-line Caption}   
  \label{fig:amdahl}
\end{figure}

When importing figures or any graphical image please verify two things:
\vspace{-0.65cm} % THE DISTANCE BETWEEN THE ":" AND THE FIRST LINE OF THE LIST MAY VARY DEPENDING ON THE TEXT LENGTH, CHANGE THIS VALUE DEPENDING ON YOUR NEEDS.
\begin{itemize} \itemsep1pt \parskip0pt \parsep0pt
\item Any number, text or symbol is in Times font and is not smaller than 
  10-point after reduction to the actual window in your paper
\item That it can be translated into PDF
\end{itemize}

Equations, such as Eq. (\ref{sample_equation}), should be centered and 
sequentially numbered to the flush right of the formula.

\begin{equation}
  \label{sample_equation}
  \mathrm{Speedup}=\frac{1}{\frac{f}{p}+(1-f)}
\end{equation}


\subsubsection{Sub-subsection level and lower: only first character uppercase}

See Table \ref{table:example} for a sample table.  The ``tabls'' package is
recommended for improved row and column spacing.  Notice the caption appears 
above the table by setting the \verb!\caption! command immediately 
after the \verb!\begin{table}!. Tables are numbered in Roman 
numerals, with the caption centered above the table, in \textbf{boldface}.  
Triple-space before and after the table.

\begin{table}[!htb]
  \centering
  \caption{\bf Parallel Performance for the Sample Problem}
  \label{table:example} 
  \begin{tabular}{|c|c|c|c|} \hline 
   Number of & Wall-Clock & Speedup & Efficiency \\
   Processors & Time$^{a}$ (min) & (T$_{s}$/T$_{p}$) & (\%) \\ \hline
    \ 1 &  100.0 & \ ---    & ---  \\ \hline
    \ 2 &   52.6 & \ 1.9    & 95.0 \\ \hline 
  \end{tabular}
\end{table}

\section{CONCLUSIONS}

Present your summary and conclusions here.

\section*{NOMENCLATURE}

If variables are extensively used in the text, a Nomenclature section would be helpful to the readers.

\section*{ACKNOWLEDGEMENTS}

This work was made possible by the 

\textbf{As an example:} The format for this template was adapted from the \LaTeX template for the PHYSOR-2018 conference posted and available on the Internet and 
most of the \LaTeX\ format definitions contained in this were already defined. The 
M\&C 2019 organizing committee deeply thank the PHYSOR-2018 technical committee 
for this great support.

% You can enter a bibliography into the document using the following format, or use the 
% bibliography style file "mandc.bst" found in the template directory.  You can use the bibliography style file
% by replacing the current bibliography block with:
% \setlength{\baselineskip}{12pt}
% \bibliographystyle{mandc}
% \bibliography{mandc}

\setlength{\baselineskip}{12pt}
\begin{thebibliography}{300}
\bibitem{journal} B. Author(s), ``Title,'' \emph{Journal Name in Italic}, 
  \textbf{Volume in Bold}, pp. 34-89 (19xx).
\bibitem{proc_paper} C. D. Author(s), ``Article Title,'' \emph{Proceedings of
  Meeting in Italic}, Location, Dates of Meeting, Vol. n, pp. 134-156 
  (19xx).
\bibitem{book} E. F. Author, \emph{Book Title in Italic}, Publisher, City \&
  Country (19xx). 
\bibitem{website} ``Spallation Neutron Source: The next-generation 
  neutron-scattering facility in the United States,'' 
  \url{http://www.sns.gov/documentation/sns\_brochure.pdf} (2002).
\end{thebibliography}

\appendix
\gdef\thesection{APPENDIX \Alph{section}}
\section{Sample Appendix 1}
\label{app:a}
If necessary, include Appendices numbered in upper case alphabetical order. This is \ref{app:a}. 



\end{document}
